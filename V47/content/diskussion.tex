\section{Diskussion}
\label{sec:Diskussion}

Für die ermittelten Debye-Temperaturen $\Theta_\text{D}$ haben sich die Werte
\begin{align*}
  \Theta_{\text{D,Exp}} &= \input{build/Theta_deb.tex} & \Theta_{\text{D,Theorie}} &= \input{build/Theta_deb_theo.tex}
\end{align*}
ergeben, was einer Abweichung des experimentellen Ergebnisses vom theoretisch errechneten Ergebnis nach dem Debye-Modell von
\begin{align*}
  \increment \Theta_\text{D} = \input{build/err_deb.tex}
\end{align*}
entspricht.
Dabei ist jedoch zu beachten, dass auch das Debye-Modell nur einer Näherung entspricht.
Dies spiegelt sich beispielsweise in der angenommenen linearen Dispersionsrelation wider.

Bei den experimentellen Werten fällt vor allem auf, dass sich die Wärmekapazität wider Erwarten nicht gänzlich monoton mit der Temperatur verhält.
Dies kann auf systematische Fehler zurückgeführt werden, welche typisch für einen thermodynamischen Versuch sind:
Trotz Vakuum sind Wärmeverluste durch Wärmeleitung, Wärmestrahlung und Konvektion nicht gänzlich auszuschließen.
Zudem musste während des Versuches der umgebende Kupferzylinder ebenfalls auf derselben Temperatur wie die Probe gehalten werden.
Da dies nicht immer perfekt möglich war, treten auch hier Beeinflussungen durch Wärmezufuhr aus der zweiten Heizspule oder durch Wärmeabgabe an den Kupferzylinder auf.
Dies könnte auch eine Erklärung für das nicht-monotone Verhalten der Wärmekapazität darstellen.

Eine Fehlerquelle bei der Auswertung besteht in dem Sachverhalt, dass zur Bestimmung der Debye-Temperatur eine Tabelle mit festen Zahlenwerten verwendet wurde.
Durch Verwendung einer genaueren Tabelle, bzw. durch numerisches Ermitteln der jeweiligen Zahlenwerte, könnte die Genauigkeiten der Daten ebenfalls verbessert werden.\\
Abschließend lässt sich sagen, dass die durch die Messung ermittelten Werte relativ gut zu der in der Theorie beschriebenen Abhängigkeit von der Temperatur passen.
