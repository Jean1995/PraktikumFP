\subsection{Durchführung}
\label{sec:durchführung}

Zunächst soll die Kupferprobe im Rezipienten abgekühlt werde.
Dazu wird der Rezipient zunächst durch die Vakuumpumpe evakuiert und danach mit Helium über das Reduzierventil geflutet.
Helium wird hier als Gas verwendet, da dieses ein deutlich besserer Wärmeleiter als Luft ist und somit die Abkühlung vereinfacht wird.
Das Dewargefäß wird mit flüssigem Stickstoff gefüllt und es wird gewartet, bis die Probe eine Temperatur von ca. $\SI{90}{\kelvin}$ erreicht hat.
Daraufhin ist der Abkühlvorgang abgeschlossen, und der Rezipient wird zur Isolierung erneut mit der Vakuumpumpe evakuiert.\\

Um der Probe
