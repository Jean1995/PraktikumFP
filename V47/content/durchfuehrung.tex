\subsection{Durchführung}
\label{sec:durchführung}

Zunächst soll die Kupferprobe im Rezipienten abgekühlt werden.
Dazu wird der Rezipient durch die Vakuumpumpe evakuiert und danach mit Helium über das Reduzierventil geflutet.
Helium wird hier als Gas verwendet, da dieses ein deutlich besserer Wärmeleiter als Luft ist und die Abkühlung somit vereinfacht wird.
Das Dewargefäß wird mit flüssigem Stickstoff gefüllt und es wird gewartet, bis die Probe eine Temperatur von ca. $\SI{90}{\kelvin}$ erreicht hat.
Daraufhin ist der Abkühlvorgang abgeschlossen und der Rezipient wird zur Isolierung erneut mit der Vakuumpumpe evakuiert.

Um der Probe nun Wärme zuzuführen, wird das Konstantstromgerät auf eine Spannung $U_{\text{Heiz}}$ und eine Stromstärke $I_{\text{Heiz}}$ eingestellt.
Während eines Messintervalles, welches durch ein vorher festgelegtes $\increment T$ definiert ist, werden Spannung und Stromstärke konstant gehalten.
Dabei wird $I_{\text{Heiz}}$ im Laufe des Versuches auf Grund der laut Theorie ansteigenden Wärmekapazität nach oben geregelt, damit die Zeitintervalle zwischen den Temperaturen nicht zu groß werden.
Die Zeiten der Messintervalle, in denen sich die Temperatur um $\increment T$ geändert hat, werden mit einem Zeitmessgerät aufgenommen sowie die genutzte Stromstärke und Spannung notiert.
Zudem wird während des gesamten Messvorgangs durch Nachregelung der Spannung und Stromstärke der Gehäuseheizung sichergestellt, dass die Temperatur des Kupferzylinders in etwa mit der Probentemperatur übereinstimmt.
Nach Erreichen einer Temperatur von ca. $\SI{300}{\kelvin}$ wird die Messung beendet.
