\section{Zielsetzung}
Im folgenden Experiment wird die molare Wärmekapazität von Kupfer bestimmt.
Hierzu wird ein Versuchsaufbau verwendet, der es ermöglicht, dem Material eine fest definierte Wärmemenge in einem bestimmten Zeitintervall hinzuzufügen.
Zudem wird aus den gemessenen Werten die Debye-Temperatur von Kupfer ermittelt.

\section{Theorie}
\label{sec:Theorie}


% 2x2 Plot
% \begin{figure*}
%     \centering
%     \begin{subfigure}[b]{0.475\textwidth}
%         \centering
%         \includegraphics[width=\textwidth]{Abbildungen/Schaltung1.pdf}
%         \caption[]%
%         {{\small Schaltung 1.}}
%         \label{fig:Schaltung1}
%     \end{subfigure}
%     \hfill
%     \begin{subfigure}[b]{0.475\textwidth}
%         \centering
%         \includegraphics[width=\textwidth]{Abbildungen/Schaltung2.pdf}
%         \caption[]%
%         {{\small Schaltung 2.}}
%         \label{fig:Schaltung2}
%     \end{subfigure}
%     \vskip\baselineskip
%     \begin{subfigure}[b]{0.475\textwidth}
%         \centering
%         \includegraphics[width=\textwidth]{Abbildungen/Schaltung4.pdf}    % Zahlen vertauscht ... -.-
%         \caption[]%
%         {{\small Schaltung 3.}}
%         \label{fig:Schaltung3}
%     \end{subfigure}
%     \quad
%     \begin{subfigure}[b]{0.475\textwidth}
%         \centering
%         \includegraphics[width=\textwidth]{Abbildungen/Schaltung3.pdf}
%         \caption[]%
%         {{\small Schaltung 4.}}
%         \label{fig:Schaltung4}
%     \end{subfigure}
%     \caption[]
%     {Ersatzschaltbilder der verschiedenen Teilaufgaben.}
%     \label{fig:Schaltungen}
% \end{figure*}

Die molare Wärmekapazität eines Materials gibt an, wie viel Wärmemenge dem Material zugeführt werden muss, um es um ein Kelvin zu erwärmen.
Es gibt drei Modelle um diese Größe zu beschreiben: das klassische, das Einstein- und das Debye-Modell.

\subsection{Das klassische Modell der molaren Wärmekapazität}

Im klassischen Modell der molaren Wärmekapazität wird davon ausgegangen, dass sich die Wärme nach dem Äquipartitionstheorem im thermischen Gleichgewicht gleichmäßig auf alle Bewegungsfreiheitsgrade verteilt.
Für die mittlere potentielle Energie eines Atoms pro Freiheitsgrad ergibt sich demnach
\begin{equation}
  \left\langle u \right\rangle_{\text{kl}} = \frac{1}{2}k_{\text{B}}T,
\end{equation}
wobei $T$ die Temperatur und $k_{\text{B}}$ die Boltzmann-Konstante \cite{scipy} sind.
Aus dem Virialsatz geht hervor, dass die mittlere kinetische Energie der mittleren potentiellen Energie entspricht.
Daher lautet die Energie eines Atoms
\begin{equation}
  U = 3k_{\text{B}}T
\end{equation}
und für ein Mol
\begin{equation}
  U = 3N_{\text{L}}k_{\text{B}}T = 3RT,
\end{equation}
mit der Loschmidtzahl $N_{\text{L}}$ und der allgemeinen Gaskonstante $R$ \cite{scipy}.
Die Molwärme bei konstantem Volumen ergibt sich zu
\begin{equation}
  C_{\text{V}} = \frac{\partial U}{\partial T}\biggr|_V = 3R.
\end{equation}
Diese Vorhersage einer temperatur- und materialunabhängigen molaren Wärmekapazität wird als Dulong-Petit'sches Gesetz bezeichnet.
Empirisch wird jedoch festgestellt, dass die Wärmekapazität bei geringen Temperaturen und geringen Atommassen einerseits kleiner als die Vorhersage und andererseits stark temperaturabhängig ist.
Erst bei hinreichend großen Temperaturen wird der Wert des Dulong-Petit'sches Gesetzes asymptotisch erreicht.
Diese Diskrepanz wird mit den folgenden Modellen korrigiert.

\subsection{Das Einsteinmodell}

Im Einsteinmodell wird beachtet, dass die atomaren Oszillatoren nur Energien in Einheiten von $\hbar\omega$ austauschen können.
Zudem wird angenommen, dass alle Atome einheitlich mit derselben Frequenz $\omega$ schwingen.
Aus der Boltzmann-Verteilung
\begin{equation}
  W(n) = e^{-\frac{n\hbar\omega}{k_{\text{B}}T}}
\end{equation}
welche angibt, mit welcher Wahrscheinlichkeit ein Oszillator, der bei der Temperatur $T$ im Gleichgewicht mit seiner Umgebung steht, die Energie $n\hbar\omega$ besitzt, kann die mittlere Energie eines Oszillators pro Freiheitsgrad über die Formel
\begin{equation}
  \left\langle u \right\rangle_{\text{E}} = \frac{\sum_{n=0}^{\infty}n\hbar\omega e^{-\frac{n\hbar\omega}{k_{\text{B}}T}}}{\sum_{n=0}^{\infty} e^{-\frac{n\hbar\omega}{k_{\text{B}}T}}}
\end{equation}
errechnet werden.
Hierbei wird die Summe über alle Energien multipliziert mit deren Wahrscheinlichkeit auf die Summe aller Wahrscheinlichkeiten normiert.
Das führt zur Energie eines Mols von
\begin{equation}
  U = 3N_{\text{L}}\frac{\hbar\omega}{e^{\frac{\hbar\omega}{k_{\text{B}}T}}-1}.
\end{equation}
Daraus folgt, dass die Molwärme
\begin{equation}
  C_{\text{V}} = 3R\frac{\hbar^2\omega^2}{k_{\text{B}}^2}\frac{1}{T^2}\frac{e^{\frac{\hbar\omega}{k_{\text{B}}T}}}{\left(e^{\frac{\hbar\omega}{k_{\text{B}}T}}-1\right)^2}
\end{equation}
beträgt.
Die Annahme, dass alle Atome mit derselben Frequenz schwingen, ist jedoch eine grobe Näherung. Diese wird im Debye-Modell korrigiert.

\subsection{Das Debye-Modell}

Das Debye-Modell zieht die spektrale Verteilung $Z(\omega)$ der Eigenschwingungen mit ein, woraus sich die innere Energie aus dem Integral über alle Eigenschwingungen ergibt.
Für die Molwärme folgt der Ausdruck
\begin{equation}
  C_{\text{V}} = \frac{\mathrm{d}}{\mathrm{d}T}\int_{0}^{\omega_{\text{max}}}Z(\omega)\frac{\hbar\omega}{e^{\frac{\hbar\omega}{k_{\text{B}}T}}-1}d\omega.
\end{equation}
Die Funktion $Z(\omega)$ wird insofern angenähert, dass von einem isotropen Kristall mit einer linearen Dispersionsrelation ausgegangen wird.
Daraus folgt, dass die Phasengeschwindigkeit einer elastischen Welle im Material nicht von der Ausbreitungsrichtung oder Frequenz abhängt.
Diese Näherung, welche es erlaubt die spektrale Verteilung durch Abzählung der Eigenschwingungen in einem Würfel der Kantenlänge $L$ im Frequenzintervall $[\omega,\omega+d\omega]$ zu berechnen, wird Debye-Modell genannt.
Dies führt zu einer spektralen Dichte von
\begin{equation}
  Z(\omega)d\omega = \frac{3L^3}{2\pi^2}\omega^2\left(\frac{1}{v_l^3}+\frac{2}{v_{tr}^3}\right)d\omega. \label{eqn:1}
\end{equation}
Hierbei bezeichnen $v_l$ und $v_{tr}$ die longitudinale und transversale Phasengeschwindigkeit.
Da ein Atom nur drei Eigenschwingungen besitzt, hat ein Kristall aus $N$ Atomen höchstens $3N$ Eigenschwingungen, woraus folgt, dass es eine obere Grenze der Frequenz geben muss, die Debye-Frequenz $\omega_{\text{D}}$, und es folgt die Relation
\begin{equation}
  \int_{0}^{\omega_{\text{D}}}Z(\omega)d\omega = 3N. \label{eqn:2}
\end{equation}
Aus \eqref{eqn:1} und \eqref{eqn:2} folgt
\begin{equation}
  \omega_{\text{D}}^3 = \frac{18\pi^2N}{L^3}\left(\frac{1}{v_l^3}+\frac{2}{v_{tr}^3} \right)^{-1}. \label{eqn:3}
\end{equation}
Demnach vereinfacht sich die spektrale Dichte zu
\begin{equation}
  Z(\omega)d\omega = \frac{3N}{\omega^3}\omega^2d\omega.
\end{equation}
Die Molwärme nach Debye beträgt somit
\begin{equation}
  C_{\text{V}} = \frac{\mathrm{d}}{\mathrm{d}T}\frac{9N\hbar}{\omega_{\text{D}}^3} \int_{0}^{\omega_{\text{D}}}\frac{\omega^3}{e^{\frac{\hbar\omega}{k_{\text{B}}T}}-1}d\omega.
\end{equation}
Mit Hilfe der Substitutionen
\begin{equation}
  x := \frac{\hbar\omega}{k_{\text{B}}T} \:\: \text{und} \:\: \frac{\Theta_{\text{D}}}{T} := \frac{\hbar\omega_{\text{D}}}{k_{\text{B}}T} \label{eqn:4}
\end{equation}
lässt sich die Molwärme schreiben als
\begin{equation}
  C_{\text{V}} = \frac{\mathrm{d}}{\mathrm{d}T} \left( 9N k_{\text{B}}T \left(\frac{T}{\Theta_{\text{D}}}\right)^3 \int_{0}^{\frac{\omega_{\text{D}}}{T}}\frac{x^3}{e^x-1}dx\right).
\end{equation}
Durch Nutzen der Leibniz-Regel für Parameterintegrale lässt sich diese Formel auf
\begin{equation}
  C_{\text{V}} = 9R \left(\frac{T}{\Theta_{\text{D}}}\right)^3 \int_{0}^{\frac{\Theta_{\text{D}}}{T}} \frac{x^4e^x}{\left(e^x-1\right)^2}dx
\end{equation}
vereinfachen.
Hierbei ist $\Theta_{\text{D}}$ die materialabhängige Debye-Temperatur, die Funktion $C_{\text{V}}\left(\frac{\Theta_{\text{D}}}{T}\right)$ hingegen ist universell.
Ihre Werte können im Anhang eingesehen werden.
%\ref{tabelle}
Die Molwärme nach Debye zeigt dasselbe asymptotische Verhalten wie die des Einstein-Modells.
Im Tieftemperaturbereich ist jene jedoch proportional zu $T^3$, die nach Einstein steigt exponentiell an.\\
Das Debye-Modell ist aber auch nur eine Näherung.
Um exakte Ergebnisse zu erhalten, müssen die wirklichen Dispersionsrelationen der Longitudinal- und Transversalwellen berücksichtigt werden.
Das Einstein- und Debyemodell vernachlässigen jedoch beide den Beitrag von Leitungselektronen zur Molwärme, welcher bei tiefen Temperaturen einen Beitrag proportional zu $T$ liefert \cite{skript}.
