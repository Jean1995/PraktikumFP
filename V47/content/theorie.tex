\section{Theorie}
\label{sec:Theorie}

\cite{sample}

% 2x2 Plot
% \begin{figure*}
%     \centering
%     \begin{subfigure}[b]{0.475\textwidth}
%         \centering
%         \includegraphics[width=\textwidth]{Abbildungen/Schaltung1.pdf}
%         \caption[]%
%         {{\small Schaltung 1.}}
%         \label{fig:Schaltung1}
%     \end{subfigure}
%     \hfill
%     \begin{subfigure}[b]{0.475\textwidth}
%         \centering
%         \includegraphics[width=\textwidth]{Abbildungen/Schaltung2.pdf}
%         \caption[]%
%         {{\small Schaltung 2.}}
%         \label{fig:Schaltung2}
%     \end{subfigure}
%     \vskip\baselineskip
%     \begin{subfigure}[b]{0.475\textwidth}
%         \centering
%         \includegraphics[width=\textwidth]{Abbildungen/Schaltung4.pdf}    % Zahlen vertauscht ... -.-
%         \caption[]%
%         {{\small Schaltung 3.}}
%         \label{fig:Schaltung3}
%     \end{subfigure}
%     \quad
%     \begin{subfigure}[b]{0.475\textwidth}
%         \centering
%         \includegraphics[width=\textwidth]{Abbildungen/Schaltung3.pdf}
%         \caption[]%
%         {{\small Schaltung 4.}}
%         \label{fig:Schaltung4}
%     \end{subfigure}
%     \caption[]
%     {Ersatzschaltbilder der verschiedenen Teilaufgaben.}
%     \label{fig:Schaltungen}
% \end{figure*}

Die Molwärme eines Materials gibt an, wieviel Wärmemenge dem Material zugeführt werden muss, um es um einen Grad Kelvin zu erwärmen.
Es gibt drei Ansätze diese Größe zu beschreiben: Das klassische, das Einstein- und das Debye-Modell.

\subsection{Das klassische Modell der Molwärme}

Im klassischen Modell der Molwärme wird davon ausgegangen, dass sich die Wärme nach dem Äquipartitionstheorem im thermischen Gleichgewicht gleichmäßig auf alle Bewegungsfreiheitsgrade verteilt.
Für die mittlere potentielle Energie eines Atoms pro Freiheitsgrad ergibt sich demnach
\begin{equation}
  <u>_{kl} = \frac{1}{2}k_BT,
\end{equation}
wobei $T$ die Temperatur und $k_B$ die Boltzmann-Konstante \cite{scipy} ist.
Aus dem Virialsatz geht hervor, dass die mittlere kinetische Energie der mittleren potentiellen Energie entspricht.
Daher lautet die Energie eines Atoms
\begin{equation}
  U = 3k_BT
\end{equation}
und für ein Mol
\begin{equation}
  U = 3N_Lk_BT = 3RT,
\end{equation}
mit der Loschmidtzahl $N_L$ und der allgemeinen Gaskonstante $R$. \cite{scipy}
Die Molwärme bei konstantem Volumen ergibt sich zu
\begin{equation}
  C_V = \frac{\partial U}{\partial T}\bigr|_V = 3R.
\end{equation}
Diese ist temperatur- und materialunabhängig und genügt dem Dulong-Petit'schen Gesetz.
Empirisch wird diese Annahme jedoch wiederlegt.
Bei hinreichend großen Temperaturen wird dieser Wert asymptotisch erreicht.

\subsection{Das Einsteinmodell}

Im Einsteinmodell wird beachtet, dass die atomaren Oszillatoren nur Energien in Einheiten von $\hbar\omega$ austauschen können.
Zudem wird angenommen, dass alle Atome einheitlich mit der selben Frequenz $\omega$ schwingen.
Aus der Boltzmann-Verteilung
\begin{euqation}
  W(n) = e^{-\frac{n\hbar\omega}{k_BT}}
\end{euqation}
welche angibt, mit welcher Wahrscheinlichkeit ein Oszillator, der bei der Temperatur $T$ im Gleichgewicht mit seiner Umgebung steht, die Energie $n\hbar\omega$ besitzt, kann die mittlere Energie eines Oszillators pro Freiheitsgrad über die Formel
\begin{equation}
  <u>_E = \frac{\sum_{n=0}^{\infty}n\hbar\omega e^{-\frac{n\hbar\omega}{k_BT}}}{\sum_{n=0}^{\infty} e^{-\frac{n\hbar\omega}{k_BT}}}
\end{euqation}
errechnet werden.
Hierbei wird die Summe über alle Energien multipliziert mit deren Wahrscheinlichkeit auf die Summe aller Wahrscheinlichkeiten normiert.
Das führt zur Energie eines Mols von
\begin{euqation}
  U = 3N_L\frac{\hbar\omega}{e^{\frac{n\hbar\omega}{k_BT}}-1}
\end{equation}
und für ein Mol
