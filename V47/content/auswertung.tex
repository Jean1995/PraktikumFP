\section{Auswertung}
\label{sec:Auswertung}

% % Examples
% \begin{equation}
%   U(t) = a \sin(b t + c) + d
% \end{equation}
%
% \begin{align}
%   a &= \input{build/a.tex} \\
%   b &= \input{build/b.tex} \\
%   c &= \input{build/c.tex} \\
%   d &= \input{build/d.tex} .
% \end{align}
% Die Messdaten und das Ergebnis des Fits sind in Abbildung~\ref{fig:plot} geplottet.
%
% %Tabelle mit Messdaten
% \begin{table}
%   \centering
%   \caption{Messdaten.}
%   \label{tab:data}
%   \sisetup{parse-numbers=false}
%   \begin{tabular}{
% % format 1.3 bedeutet eine Stelle vorm Komma, 3 danach
%     S[table-format=1.3]
%     S[table-format=-1.2]
%     @{${}\pm{}$}
%     S[table-format=1.2]
%     @{\hspace*{3em}\hspace*{\tabcolsep}}
%     S[table-format=1.3]
%     S[table-format=-1.2]
%     @{${}\pm{}$}
%     S[table-format=1.2]
%   }
%     \toprule
%     {$t \:/\: \si{\milli\second}$} & \multicolumn{2}{c}{$U \:/\: \si{\kilo\volt}$\hspace*{3em}} &
%     {$t \:/\: \si{\milli\second}$} & \multicolumn{2}{c}{$U \:/\: \si{\kilo\volt}$} \\
%     \midrule
%     \input{build/table.tex}
%     \bottomrule
%   \end{tabular}
% \end{table}
%
% % Standard Plot
% \begin{figure}
%   \centering
%   \includegraphics{build/plot.pdf}
%   \caption{Messdaten und Fitergebnis.}
%   \label{fig:plot}
% \end{figure}
%
% 2x2 Plot
% \begin{figure*}
%     \centering
%     \begin{subfigure}[b]{0.475\textwidth}
%         \centering
%         \includegraphics[width=\textwidth]{Abbildungen/Schaltung1.pdf}
%         \caption[]%
%         {{\small Schaltung 1.}}
%         \label{fig:Schaltung1}
%     \end{subfigure}
%     \hfill
%     \begin{subfigure}[b]{0.475\textwidth}
%         \centering
%         \includegraphics[width=\textwidth]{Abbildungen/Schaltung2.pdf}
%         \caption[]%
%         {{\small Schaltung 2.}}
%         \label{fig:Schaltung2}
%     \end{subfigure}
%     \vskip\baselineskip
%     \begin{subfigure}[b]{0.475\textwidth}
%         \centering
%         \includegraphics[width=\textwidth]{Abbildungen/Schaltung4.pdf}    % Zahlen vertauscht ... -.-
%         \caption[]%
%         {{\small Schaltung 3.}}
%         \label{fig:Schaltung3}
%     \end{subfigure}
%     \quad
%     \begin{subfigure}[b]{0.475\textwidth}
%         \centering
%         \includegraphics[width=\textwidth]{Abbildungen/Schaltung3.pdf}
%         \caption[]%
%         {{\small Schaltung 4.}}
%         \label{fig:Schaltung4}
%     \end{subfigure}
%     \caption[]
%     {Ersatzschaltbilder der verschiedenen Teilaufgaben.}
%     \label{fig:Schaltungen}
% \end{figure*}
Die untersuchte Kupferprobe hat eine Masse von $m = \SI{342}{\gram}$ \cite{skript}.
In Tabelle \ref{tab:eig} sind die Eigenschaften von Kupfer dargestellt.

%\input{build/Tabelle_eigenschaften.tex}

Dabei beschreibt $\rho$ die Dichte, $M$ die Molmasse, $\kappa$ den Kompressionsmodul und $V_0$ das Molvolumen.
Die aufgenommenen Messwerte sind in Tabelle \ref{tab:1} wiedergegeben.
Hierbei ist $R$ der Thermowiderstand an der Probe, $t$ die Messzeit, $U$ und $I$ beschreiben jeweils die Spannung und die Stromstärke der Heizwindung.
Das Ohmmeter hat eine Ungenauigkeit von $\delta R = \SI{1}{\ohm}$, das Voltmeter eine von $\delta U = \SI{0.1}{\volt}$, sowie das Amperemeter eine von $\delta I = \SI{0.01}{\ampere}$.
Die Zeit wird eine Ableseungenauigkeit von $\delta t = \SI{1}{\second}$ zugewiesen.\\
Die Temperatur $T$ in Kelvin kann aus dem Thermowiderstand über die Formel
\begin{equation}
  T = 0.00134R^2+2.296R+30.13
\end{equation}
gewonnen werden.
\input{build/Tabelle_messwerte.tex}

Die der Probe zugeführte Energie $\Delta E$ wird aus dem Produkt aus Spannung, Stromstärke und Aufheizzeit $\Delta t$,
\begin{equation}
  \Delta E =  U\cdot I\cdot \Delta t
\end{equation}
hergeleitet.\\
Dabei werden jeweils die Messwerte zu Beginn der Heizperiode genommen.
Für die Wärmekapazität bei konstantem Druck folgt
\begin{equation}
  C_{\text{p}} = \frac{M}{m}\frac{\Delta E}{\Delta T} = \frac{U\cdot I\cdot \Delta t}{\Delta T}.
\end{equation}
Daraus resultiert einschließlich der Messwerte die Tabelle \ref{tab:3}.

\input{build/Tabelle_cp.tex}





\input{build/Tabelle_ausdehnung.tex}
