\section{Auswertung}
\label{sec:Auswertung}

% % Examples
% \begin{equation}
%   U(t) = a \sin(b t + c) + d
% \end{equation}
%
% \begin{align}
%   a &= \input{build/a.tex} \\
%   b &= \input{build/b.tex} \\
%   c &= \input{build/c.tex} \\
%   d &= \input{build/d.tex} .
% \end{align}
% Die Messdaten und das Ergebnis des Fits sind in Abbildung~\ref{fig:plot} geplottet.
%
% %Tabelle mit Messdaten
% \begin{table}
%   \centering
%   \caption{Messdaten.}
%   \label{tab:data}
%   \sisetup{parse-numbers=false}
%   \begin{tabular}{
% % format 1.3 bedeutet eine Stelle vorm Komma, 3 danach
%     S[table-format=1.3]
%     S[table-format=-1.2]
%     @{${}\pm{}$}
%     S[table-format=1.2]
%     @{\hspace*{3em}\hspace*{\tabcolsep}}
%     S[table-format=1.3]
%     S[table-format=-1.2]
%     @{${}\pm{}$}
%     S[table-format=1.2]
%   }
%     \toprule
%     {$t \:/\: \si{\milli\second}$} & \multicolumn{2}{c}{$U \:/\: \si{\kilo\volt}$\hspace*{3em}} &
%     {$t \:/\: \si{\milli\second}$} & \multicolumn{2}{c}{$U \:/\: \si{\kilo\volt}$} \\
%     \midrule
%     \input{build/table.tex}
%     \bottomrule
%   \end{tabular}
% \end{table}
%
% % Standard Plot
% \begin{figure}
%   \centering
%   \includegraphics{build/plot.pdf}
%   \caption{Messdaten und Fitergebnis.}
%   \label{fig:plot}
% \end{figure}
%
% 2x2 Plot
% \begin{figure*}
%     \centering
%     \begin{subfigure}[b]{0.475\textwidth}
%         \centering
%         \includegraphics[width=\textwidth]{Abbildungen/Schaltung1.pdf}
%         \caption[]%
%         {{\small Schaltung 1.}}
%         \label{fig:Schaltung1}
%     \end{subfigure}
%     \hfill
%     \begin{subfigure}[b]{0.475\textwidth}
%         \centering
%         \includegraphics[width=\textwidth]{Abbildungen/Schaltung2.pdf}
%         \caption[]%
%         {{\small Schaltung 2.}}
%         \label{fig:Schaltung2}
%     \end{subfigure}
%     \vskip\baselineskip
%     \begin{subfigure}[b]{0.475\textwidth}
%         \centering
%         \includegraphics[width=\textwidth]{Abbildungen/Schaltung4.pdf}    % Zahlen vertauscht ... -.-
%         \caption[]%
%         {{\small Schaltung 3.}}
%         \label{fig:Schaltung3}
%     \end{subfigure}
%     \quad
%     \begin{subfigure}[b]{0.475\textwidth}
%         \centering
%         \includegraphics[width=\textwidth]{Abbildungen/Schaltung3.pdf}
%         \caption[]%
%         {{\small Schaltung 4.}}
%         \label{fig:Schaltung4}
%     \end{subfigure}
%     \caption[]
%     {Ersatzschaltbilder der verschiedenen Teilaufgaben.}
%     \label{fig:Schaltungen}
% \end{figure*}

\subsection{Messdaten}
Die untersuchte Kupferprobe hat eine Masse von $m = \SI{342}{\gram}$ \cite{skript}.
In Tabelle \ref{tab:eig} sind die Eigenschaften von Kupfer dargestellt.

\begin{table}
    \centering
    \caption{Materialeigenschaften von Kupfer \cite[S. 1509]{3527261699},\cite{komp},\cite{skript}. }
    \label{tab:eig}
    \sisetup{parse-numbers=false}
    \begin{tabular}{
	S[table-format=3.1]
	S[table-format=3.1]
	S[table-format=1.1]
	S[table-format=4.2]
	}
	\toprule
  {$\rho \:/\: \si{\kilo\gram\per\meter\tothe{3}}$} &
  {$M \:/\: 10^{-3}\si{\kilo\gram\per\mol}$} &  {$\kappa \:/\: 10^{9}\si{\pascal}$} &
  {$V_0 \:/\: 10^{-6}\si{\meter\tothe{3}\per\mol}$}		\\
	\midrule
    8920 & 63 & 140 & 7.06 \\

    \bottomrule
    \end{tabular}
\end{table}



%\input{build/Tabelle_eigenschaften.tex}

Dabei beschreibt $\rho$ die Dichte, $M$ die Molmasse, $\kappa$ den Kompressionsmodul und $V_0$ das Molvolumen von Kupfer.
Die aufgenommenen Messwerte sind in Tabelle \ref{tab:1} wiedergegeben.
Hierbei ist $R$ der Thermowiderstand an der Probe, $t$ die Messzeit, $U$ und $I$ beschreiben jeweils die Spannung und die Stromstärke der Heizwindung.
Als Messintervall wird ein $\increment T$ von $\SI{10}{\kelvin}$ gewählt.
Das Ohmmeter hat eine Ungenauigkeit von $\delta R = \SI{0.1}{\ohm}$, das Voltmeter eine von $\delta U = \SI{0.1}{\volt}$, sowie das Amperemeter eine von $\delta I = \SI{0.01}{\ampere}$.
Der Zeit wird eine Ableseungenauigkeit von $\delta t = \SI{1}{\second}$ zugewiesen.\\
Die Temperatur $T$ in Kelvin kann aus dem Thermowiderstand $R$ über die Formel
\begin{equation}
  T = 0.00134R^2+2.296R+30.13
\end{equation}
gewonnen werden \cite{skript}.
\input{build/Tabelle_messwerte.tex}

\newpage

\subsection{Berechnung der molaren Wärmekapazität}
Die der Probe zugeführte Energie $\Delta E$ wird aus dem Produkt aus Spannung, Stromstärke und Aufheizzeit $\Delta t$,
\begin{equation}
  \Delta E =  U\cdot I\cdot \Delta t
\end{equation}
hergeleitet.\\
Dabei werden für Spannug und Stromstärke jeweils die Messwerte zu Beginn der Heizperiode genommen.
Für die molare Wärmekapazität bei konstantem Druck folgt
\begin{equation}
  C_{\text{p}} = \frac{M}{m}\frac{\Delta E}{\Delta T} = \frac{M}{m}\frac{U\cdot I\cdot \Delta t}{\Delta T}.
\end{equation}
Daraus resultiert einschließlich der Messwerte die Tabelle \ref{tab:3}.

\input{build/Tabelle_cp.tex}

Um auf die Molwärme bei konstantem Volumen zu kommen, wird der Zusammenhang
\begin{equation}
  C_{\text{V}} = C_{\text{p}} - 9 \alpha^2 \kappa V_0 T
\end{equation}
verwendet \cite{skript}.
Hierbei entspricht $\alpha$ dem Ausdehnungskoeffizienten von Kupfer, dessen Abhängigkeit von der Temperatur in Abbildung \ref{fig:alphaa} im Anhang einzusehen ist.
Es wird jeweils die mittlere Temperatur während einer Heizperiode genutzt und somit zwischen den angegebenen Werten des Ausdehnungskoeffizienten linear interpoliert.
Die Ausdehnungskoeffizienten sowie die daraus ermittelten spezifischen Wärmekapazitäten bei konstantem Volumen sind in Tabelle \ref{tab:2} zu begutachten.

\input{build/Tabelle_ausdehnung.tex}

\subsection{Debye-Temperatur und -Frequenz}

Die Debye-Temperatur wird über die Tabelle \ref{fig:mirfallenkeinelabelmehreinfuerdenganzenmistfuckthisshitimoutwarumtueichmirdasanundesistschonhalbzweinachtsdafuq} im Anhang bestimmt, indem zu jedem $C_{\text{V}}$ der passende Tabellenwert gesucht wird.
Aus der dazugehörigen Messtemperatur wird daraus die Debye-Temperatur $\Theta_\text{D}$ ermittelt.
Es werden hier jedoch nur Temperaturen bis $\SI{170}{\kelvin}$ berücksichtigt.
In Tabelle \ref{tab:4} sind die Resultate aufgeführt.

\input{build/Tabelle_temp.tex}

Eine Mittelung der verschiedenen ermittelten Werte führt auf eine Debye-Temperatur von
\begin{align*}
  \Theta_{\text{D,gem}} &= \input{build/Theta_deb.tex}.
\end{align*}
Aus dem Zusammenhang \eqref{eqn:4} ergibt dies eine Debye-Frequenz von
\begin{align*}
  \omega_{\text{D,gem}} &= \input{build/omega_deb.tex}.
\end{align*}

Nach der Formel \eqref{eqn:3} lassen sich über die transversale und longitudinale Phasengeschwindigkeit \cite{skript} von
\begin{align*}
  v_{\text{l}}  &= \SI{4.7}{\kilo\meter\per\second},\\
  v_{\text{tr}} &= \SI{2.26}{\kilo\meter\per\second}
\end{align*}
Theoriewerte von
\begin{align*}
  \Theta_{\text{D,Theorie}} &= \input{build/Theta_deb_theo.tex},\\
  \omega_{\text{D,Theorie}} &= \input{build/omega_deb_theo.tex}
\end{align*}
berechnen.
Zuletzt sind die Gesamtergebnisse in Plot \ref{plot:1} einzusehen.

\begin{figure}
  \centering
  \includegraphics{build/fit.pdf}
  \caption{Ergebnisse für die spezifische Wärmekapazitäten und Theoriekurven, berechnet aus den experimentellen bzw. aus den theoretischen Ergebnissen für die Debye-Temperatur.}
  \label{plot:1}
\end{figure}
