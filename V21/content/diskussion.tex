\section{Diskussion}
\label{sec:Diskussion}

\subsection{Bestimmung der Vertikalkomponente des Erdmagnetfeldes}
Das Erdmagnetfeld wurde auf den Wert
\begin{align*}
  B_\text{komp,gem} &= \input{build/B_vert.tex}
\end{align*}
bestimmt.
Der Literaturwert \cite{magnet} beträgt ungefähr
\begin{align*}
  B_\text{komp,lit} &= \SI{45,07}{\micro\tesla},
\end{align*}
dies entspricht einer Abweichung von
\begin{align*}
  \delta B_\text{komp} &= \input{build/B_err}.
\end{align*}
Die Abweichung kann durch die schwierige Feineinstellung der Spule erklärt werden.%und versuch kacke

\subsection{Bestimmung der Kernspins der Isotope}
Die Kernspins der Isotope wurden zu
\begin{align*}
  I_1 &= \input{build/I_1.tex} & I_2 &= \input{build/I_2.tex}
\end{align*}
bestimmt.
Dies lässt sich gerundet mit den Kernspins $I_1 = \frac{3}{2}$ und $I_2 = \frac{5}{2}$ identifizieren \cite{iaea}.
In diesem Falle bilden sich Abweichungen von
\begin{align*}
  \delta I_1 &= \input{build/I_1_err.tex} & \delta I_2 &= \input{build/I_2_err.tex},
\end{align*}
welche relativ gering sind.

\subsection{Berechnung des Isotopenverhältnisses}
Aus dem Versuch geht ein Isotopenverhältnis von
\begin{align*}
  \frac{A_2}{A_1} &= \input{build/A_V.tex}
\end{align*}
hervor, während die Literatur ein Verhältnis von
\begin{align*}
  \frac{P\left(^{85}\text{Rb}\right)}{P\left(^{87}\text{Rb}\right)} &= \input{build/Rb_V.tex}
\end{align*}
angibt.
Der ermittelte Wert weicht daher zu
\begin{align*}
  \delta \frac{P\left(^{85}\text{Rb}\right)}{P\left(^{87}\text{Rb}\right)} &= \input{build/Rb_V_err.tex}
\end{align*}
von der Literatur ab.
Diese große Abweichung ist daher begründet, dass die Amplituden am sich durch kleinste Störungen ändernden Signalbild abgelesen werden mussten.

\subsection{Einfluss des quadratischen Zeeman-Effekts}

Die Verhältnisse wurden auf
\begin{align*}
  \left(\frac{U_\text{lin}}{U_\text{quad}}\right)_{87} &= \input{build/zeeman_1.tex} & \left(\frac{U_\text{lin}}{U_\text{quad}}\right)_{85} &= \input{build/zeeman_2.tex}
\end{align*}
bestimmt.
Somit unterscheiden sich die beiden Effekte um ca. drei Größenordnungen mit deutlich stärkerem Einfluss durch den linearen Term.
Der quadratische Zeeman-Effekt kann daher vernachlässigt werden.
