\section{Auswertung}
\label{sec:Auswertung}


\subsection{Untersuchung der Moden}

Zur Untersuchung der Moden wurden die Werte aus Tabelle \ref{tab:moden} aufgenommen.
\input{build/moden.tex}
Dabei steht $V_0$ für die Reflektorspannung, bei der das Maximum des Modus liegt, $V_1$ und $V_2$ stehen für den Anfang und das Ende der Modenkurve, $A_0$ für die Amplitude und $f$ für die dazugehörige Frequenz zum Maximum des Modus.
Der Reflektorspannung wird jeweils eine Messungenauigkeit von $\delta V_\text{Ref} = \SI{3}{\volt}$ zugeordnet und der Frequenz eine Ungenauigkeit von $\delta f = \SI{1}{\mega\hertz}$.
Aus diesen Werten resultiert das Modus-Diagramm aus Plot \ref{plot:modus}.
\begin{figure}
  \centering
  \includegraphics[height=8cm]{build/moden.pdf}
  \caption{Modus-Diagramm über die ersten drei Moden.}
  \label{plot:modus}
\end{figure}

Die Frequenzen und Spannungen halber Leistung des ersten Modus zur Berechnung der elektronischen Bandbreite und Abstimm-Empfindlichkeit sind in Tabelle \ref{tab:breite} angegeben.
\input{build/breite.tex}
Die elektronische Bandbreite berechnet sich nach
\begin{equation}
  B = f_2-f_1
\end{equation}
zu
\begin{align*}
  B &= \input{B.tex}.
\end{align*}
Die Abstimm-Empfindlichkeit wird über
\begin{equation}
  A = \frac{f_2-f_1}{V_2-V_1}
\end{equation}
zu
\begin{align*}
  A &= \input{A.tex}
\end{align*}
bestimmt.

\subsection{Bestimmung der Frequenz aus der Wellenlängenmessung}

Die Minima des Signals zur Bestimmung der Wellenlänge $\lambda_g$ im luftgefüllten Hohlleiter liegen bei
\begin{align*}
  \text{r}_1 &= \SI{115.9}{\milli\meter}, \\
  \text{r}_2 &= \SI{90.8}{\milli\meter},
\end{align*}
wobei die Lage des Schlittens auf $\SI{0.1}{\milli\meter}$ genau bestimmt werden kann.
Das Doppelte der Differenz,
\begin{align*}
  \lambda_g &= \input{lambda_g.tex},
\end{align*}
entspricht dabei der Wellenlänge $\lambda_g$.

Aus dieser lässt sich nach Formel \ref{eqn:2} die Frequenz des Hohlleiters mit einer Breite von $a = \SI{22.860 \pm 0.046}{\milli\meter}$ zu
\begin{align*}
  f &= \input{build/f.tex}
\end{align*}
bestimmen.

\subsection{Bestimmung der Dämpfungskurve}
Die Messergebnisse zur Bestimmung der Dämpfungskurve sind in Tabelle \ref{tab:daempfung} wiedergegeben.
Die Ungenauigkeit der Mikrometerschraube wird zu $\SI{0.01}{\milli\meter}$ geschätzt und kann deshalb im Plot vernachlässigt werden.
\input{build/daempfung.tex}
Die diskreten Werte für die Dämpfung werden in Plot \ref{plot:daempfung} an eine allgemeine Exponentialfunktion gefittet.
\begin{figure}
  \centering
  \includegraphics[height=8cm]{build/daempfung.pdf}
  \caption{Dämpfungskurve.}
  \label{plot:daempfung}
\end{figure}

\subsection{Bestimmung der Welligkeit}
Die Messergebnisse bezüglich des direkten Ablesens der Welligkeit sind in Tabelle \ref{tab:welligkeit} einzusehen, wobei der Wert bei der Sondentiefe von $\SI{9}{\milli\meter}$ nicht hinreichend genau abgelesen werden konnte.
\input{build/welligkeit.tex}
Für die Sondentiefe $\SI{9}{\milli\meter}$ lauten die Schlittenpositionen für die 3db-Methode
\begin{align*}
  d_1 &= \SI{112.7}{\milli\meter} \: \text{und}\\
  d_2 &= \SI{110.9}{\milli\meter}.
\end{align*}
Aus der Formel \ref{eqn:3} berechnet sich die Welligkeit zu
\begin{align*}
  S &= \input{build/S_3db.tex}.
\end{align*}
Zuletzt liefert die Abschwächermethode Dämpfungen von
\begin{align*}
  A_1 &= \SI{20}{\decibel} \: \text{und}\\
  A_2 &= \SI{38}{\decibel},
\end{align*}
für die Sondentiefe $\SI{9}{\milli\meter}$, wobei diesen Werten jeweils eine Ungenauigkeit von $\SI{1}{\decibel}$ zugeordnet wird.
Dies führt nach Formel \ref{eqn:4} auf eine Welligkeit von
\begin{align*}
  S &= \input{build/S_abs.tex}.
\end{align*}




% % Examples
% \begin{equation}
%   U(t) = a \sin(b t + c) + d
% \end{equation}
%
% \begin{align}
%   a &= \input{build/a.tex} \\
%   b &= \input{build/b.tex} \\
%   c &= \input{build/c.tex} \\
%   d &= \input{build/d.tex} .
% \end{align}
% Die Messdaten und das Ergebnis des Fits sind in Abbildung~\ref{fig:plot} geplottet.
%
% %Tabelle mit Messdaten
% \begin{table}
%   \centering
%   \caption{Messdaten.}
%   \label{tab:data}
%   \sisetup{parse-numbers=false}
%   \begin{tabular}{
% % format 1.3 bedeutet eine Stelle vorm Komma, 3 danach
%     S[table-format=1.3]
%     S[table-format=-1.2]
%     @{${}\pm{}$}
%     S[table-format=1.2]
%     @{\hspace*{3em}\hspace*{\tabcolsep}}
%     S[table-format=1.3]
%     S[table-format=-1.2]
%     @{${}\pm{}$}
%     S[table-format=1.2]
%   }
%     \toprule
%     {$t \:/\: \si{\milli\second}$} & \multicolumn{2}{c}{$U \:/\: \si{\kilo\volt}$\hspace*{3em}} &
%     {$t \:/\: \si{\milli\second}$} & \multicolumn{2}{c}{$U \:/\: \si{\kilo\volt}$} \\
%     \midrule
%     \input{build/table.tex}
%     \bottomrule
%   \end{tabular}
% \end{table}
%
% % Standard Plot
% \begin{figure}
%   \centering
%   \includegraphics{build/plot.pdf}
%   \caption{Messdaten und Fitergebnis.}
%   \label{fig:plot}
% \end{figure}
%
% 2x2 Plot
% \begin{figure*}
%     \centering
%     \begin{subfigure}[b]{0.475\textwidth}
%         \centering
%         \includegraphics[width=\textwidth]{Abbildungen/Schaltung1.pdf}
%         \caption[]%
%         {{\small Schaltung 1.}}
%         \label{fig:Schaltung1}
%     \end{subfigure}
%     \hfill
%     \begin{subfigure}[b]{0.475\textwidth}
%         \centering
%         \includegraphics[width=\textwidth]{Abbildungen/Schaltung2.pdf}
%         \caption[]%
%         {{\small Schaltung 2.}}
%         \label{fig:Schaltung2}
%     \end{subfigure}
%     \vskip\baselineskip
%     \begin{subfigure}[b]{0.475\textwidth}
%         \centering
%         \includegraphics[width=\textwidth]{Abbildungen/Schaltung4.pdf}    % Zahlen vertauscht ... -.-
%         \caption[]%
%         {{\small Schaltung 3.}}
%         \label{fig:Schaltung3}
%     \end{subfigure}
%     \quad
%     \begin{subfigure}[b]{0.475\textwidth}
%         \centering
%         \includegraphics[width=\textwidth]{Abbildungen/Schaltung3.pdf}
%         \caption[]%
%         {{\small Schaltung 4.}}
%         \label{fig:Schaltung4}
%     \end{subfigure}
%     \caption[]
%     {Ersatzschaltbilder der verschiedenen Teilaufgaben.}
%     \label{fig:Schaltungen}
% \end{figure*}
