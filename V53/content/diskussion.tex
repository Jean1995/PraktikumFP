\section{Diskussion}
\label{sec:Diskussion}

\subsection{Modus-Diagramm und elektrische Kenngrößen}
Bezüglich des Modus-Diagramms in Plot \ref{plot:modus} fällt auf, dass der erste Modus deutlich kleiner ist als der zweite.
Der Theorie nach müsste der erste Modus jedoch das größte Maximum besitzen, da die Elektronen in dieser Einstellung die größte kinetische Energie besitzen aufgrund der größten Reflektorspannung.
Eine Erklärung könnte sein, dass bei einer so großen Reflektorspannung manche Elektronen, die den Glühdraht mit nicht hinreichend großer Energie verlassen, den Resonator aufgrund der hohen Abstoßung nicht mehr erreichen können.
Die Intensität wäre daher abgeschwächt.\\
Der sonstige Verlauf der Moden bestätigt die Theorie, obgleich die aufgenommenen Kurven nicht ganz symmetrisch sind.
Dies ist jedoch weniger ein Ablesefehler, als mehr ein systematischer Fehler gegeben durch den Versuchsaufbau.\\
Bei der Bestimmung der Abstimm-Empfindlichkeit,
\begin{align*}
  A &= \input{build/A.tex},
\end{align*}
ist ein systematischer Fehler beim Ablesen der Reflektorspannungen halber Leistung unterlaufen, da der Wert der linken Seite $V_\text{Ref} = \SI{205}{\volt}$ dem Anfangswert des Modus entspricht.
Dies heißt, der Wert halber Leistung entspräche dem Wert der Nullleistung.
Stattdessen wäre ein Wert im Intervall zwischen $\SI{205}{\volt}$ und $\SI{215}{\volt}$ zu erwarten gewesen.
\subsection{Frequenzmessung}
Die ermittelten Werte für die Frequenzmessung lauten
\begin{align*}
  f_\text{gem} &= \input{build/f.tex}, \\
  f_\text{app} &= \SI{9000 \pm 1}{\mega\hertz}.
\end{align*}
Dies entspricht einer Abweichung von
\begin{align*}
  \delta f = \input{build/abw_f.tex}.
\end{align*}
Diese Abweichung ist relativ klein, jedoch ergibt sich keine Überschneidung der Fehlerintervalle.
\subsection{Dämpfungskurve}
Die aufgenommene Dämpfungskurve \ref{plot:daempfung} entspricht in guter Näherung der wahren mit einem konstanten Offset.
Dies lässt sich erneut nur durch einen systematischen Fehler im Aufbau erklären.
Mögliche Erklärungen wären eine falsch geeichte Skala an der Schraube des Dämpfungsgliedes oder ein nicht berücksichtigter Offset im SWR-Meter
\subsection{Welligkeitsmessung}
Bei der direkten Messung der Welligkeit konnte der Wert für eine Sondentiefe von $\SI{9}{\milli\meter}$ aufgrund der logarithmischen Skala sehr schlecht abgelesen werden, daher kann dieser Messung keine Wertigkeit zugeordnet werden.
Die 3db- und Abschwächermethode liefern dafür eine Welligkeit von
\begin{align*}
  S_{3\text{db}} = \input{build/S_3db.tex} \:\: \text{und}& \:\: S_\text{abs} = \input{build/S_abs.tex}.
\end{align*}
Dabei weist die 3db-Methode den kleineren Fehler auf.
Der Unterschied der Fehler ist jedoch nicht signifikant genug, daher wird willkürlich die relative Abweichung von der Abschwächer- zur 3db-Methode berechnet.
Sie beträgt
\begin{align*}
  \delta S = \input{build/abw_S.tex}.
\end{align*}
Diese Abweichung ist sehr klein; außerdem schneiden sich beide Fehlerintervalle.\\
Insgesamt weist dieser Versuch einige systematische Fehler auf, welche durch Ausschluss von Ablesefehlern nur durch den Aufbau erklärt werden können.
Die 3db-Methode und die Abschwächer-Methode liefern beide unabhängig ähnliche Werte für die Welligkeit, daher kann nicht entschieden werden, welche Methode besser ist.
