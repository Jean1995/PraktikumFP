\section{Diskussion}
\label{sec:Diskussion}

Die in diesem Versuch ermittelte mittlere Lebensdauer beträgt
\begin{align*}
  \tau = \input{tau.tex}.
\end{align*}
In der Literatur \cite{Agashe:2014kda} ist die mittlere Lebensdauer eines Myons als
\begin{align*}
  \tau = \input{tau_lit.tex}
\end{align*}
angegeben.
Dies entspricht einer Abweichung des Messwertes zum Theoriewert von
\begin{align*}
  \Delta \tau = \input{tau_lit_abw.tex}.
\end{align*}
Der Theoriewert liegt somit innerhalb der $2\sigma$-Umgbung des berechneten Messergebnisses.\\
Als Fehlerquelle ist zunächst zu nennen, dass es sich bei dem Zerfall der Myonen um einen statistischen Prozess handelt.
Trotz einer relativ langen Messdauer hat sich noch keine eindeutige Exponentialverteilung eingestellt und die statistischen Abweichungen in den Messraten überwiegen.
Dieses Problem ließe sich durch eine deutlich größere Messzeit beheben.
Ein weiterer Fehler ist, dass negativ geladene Myonen von Atomkernen unter Bildung von myonischen Atomen eingefangen werden können.
Dieser Effekt verfälscht die Messergebnisse.
Zusätzlich wird am Fit deutlich, dass die Fitfunktion insbesondere für geringe Lebensdauern die hohen Zählraten nicht optimal trifft.
Eine bessere Kanalauflösung im Bereich geringerer Lebensdauer oder eine größere Anzahl an Messwerten könnte einen besseren Fit ermöglichen.
Des Weiteren kann trotz Nutzung der Diskriminatoren und Koinzidenz nicht vollständig ausgeschlossen werden, dass Fehlmessungen durch thermische Emission auftritt oder wahre Ereignisse verworfen werden.
Ein Hinweis darauf, dass dies geschieht, ist, dass der durch den Fit bestimmte Untergrund
\begin{align*}
  U_0 = \input{build/U_val.tex}
\end{align*}
größer ist, als der theoretisch bestimmte, welcher die thermische Emission von Elektronen nicht berücksichtigt,
\begin{align*}
  U_{\text{Theorie}} = \input{build/U_theo.tex}.
\end{align*}
Der Fehler des durch den Fit bestimmten Wertes ist jedoch so groß, dass dies kein aussagekräftiges Argument ist.
