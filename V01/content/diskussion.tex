\section{Diskussion}
\label{sec:Diskussion}

Die Auflösungszeit der Koinzidenz beträgt
\begin{align*}
  \Delta t_\text{K} = \SI{19}{\nano\second}.
\end{align*}
Dieser Wert ist eine sehr grobe Abschätzung aus der Abbildung \ref{plot:verzoegerung}.
Für eine genauere Bestimmung hätten mehr Werte im Bereich größerer Verzögerungen aufgenommen werden sollen.
Dadurch, dass zusätzlich die Breiten der Diskriminatorimpulse bei den Messergebnissen fehlen, kann der Wert von $\Delta t_\text{K}$ auch nicht in Verbindung der Breiten gesetzt werden.
Qualitativ wäre zu erwarten gewesen, dass ein breiterer Diskriminatorimpuls dazu führt, dass das Plateau in Abbildung \ref{plot:verzoegerung} breiter wird und somit eine höhere Halbwertsbreite auftritt. %<- Stay or go?

Die in diesem Versuch ermittelte mittlere Lebensdauer beträgt laut \eqref{eqn:result}
\begin{align*}
  \tau = \input{tau_gew.tex}.
\end{align*}
In der Literatur \cite{Agashe:2014kda} ist die mittlere Lebensdauer eines Myons als
\begin{align*}
  \tau = \input{tau_lit.tex}
\end{align*}
angegeben.
Dies entspricht einer Abweichung des Messwertes zum Theoriewert von
\begin{align*}
  \Delta \tau = \input{tau_lit_abw.tex}.
\end{align*}
%Der Theoriewert liegt somit innerhalb der $2\sigma$-Umgebung des berechneten Messergebnisses.\\
Die absolute Abweichung lässt sich unter anderem dadurch erklären, dass es sich bei dem Zerfall der Myonen um einen statistischen Prozess handelt.
Trotz einer relativ langen Messdauer hat sich noch keine eindeutige Exponentialverteilung eingestellt und die statistischen Abweichungen in den Messraten überwiegen.
Dies ist an Abbildung \ref{plot:punkte_err_fit_robert} zu erkennen, da der Fit viele Fehlerbalken nicht schneidet.
Dieses Problem ließe sich durch eine deutlich größere Messzeit beheben.\\
%Ein weiterer Fehler ist, dass negativ geladene Myonen von Atomkernen unter Bildung von myonischen Atomen eingefangen werden können.
%Dieser Effekt verfälscht die Messergebnisse.\\

Zusätzlich wird am Fit deutlich, dass die Fitfunktion insbesondere für geringe Lebensdauern die hohen Zählraten nicht optimal trifft.
Es steht generell offen, ob die beiden Messwerte im Bereich kleiner Lebensdauern mit sehr hohen Zählraten durch einen systematischen Fehler ausgelöst wurden. % <- Geändert
%Sie wurden berücksichtigt, da sie den Trend der Theorie unterstützen. %<- Wieso schreibst du das? Hört sich schlecht an.
Eine bessere Kanalauflösung im Bereich geringerer Lebensdauer oder eine größere Anzahl an Messwerten könnte einen besseren Fit bzw. einen besseren Einblick im Bereich kleiner Lebensdauern ermöglichen.
Des Weiteren kann trotz Nutzung der Diskriminatoren und Koinzidenz nicht vollständig ausgeschlossen werden, dass Fehlmessungen durch thermische Emission auftritt. %oder wahre Ereignisse verworfen werden. <- Sollen wir den Satz wirklich reinnehmen?? Wir hatten ja schon darüber gesprochen dass es unwahrscheinlich ist, dass gleichzeitig thermische EMission auftritt und ein Myon durchfliegt und somit den Stopimpuls auflöst (oder anders herum)
Ein Hinweis darauf, dass dies geschieht, ist, dass der durch den Fit bestimmte Untergrund
\begin{align*}
  U_0 = \input{build/U_val_gew.tex}
\end{align*}
wesentlich kleiner ist, als der theoretisch bestimmte, welcher die thermische Emission von Elektronen durch die gemessenen Startsignale berücksichtigt,
\begin{align*}
  U_{\text{Theorie}} = \input{build/U_theo.tex}.
\end{align*}
Demnach konnten die Fehlimpulse der thermischen Emission durch die Koinzidenz und die Diskriminatoren nicht wirksam herausgefiltert werden.
Der große Theoriewert könnte auch dadurch resultieren, dass negativ geladene Myonen von Atomkernen unter Bildung von myonischen Atomen eingefangen werden können und daher auch ein Start- aber kein Stoppimpuls entsteht.
%Der Fehler des durch den Fit bestimmten Wertes ist jedoch so groß, dass dies kein aussagekräftiges Argument ist.
