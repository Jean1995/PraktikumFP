\section{Auswertung}
\label{sec:Auswertung}

\subsection{Kalibration des Vielkanalysators}

Mit Hilfe von Doppelimpulsen, deren Abstand in $\SI{1}{\micro\second}$-Schritten vergrößert wird, konnte durch eine lineare Fitfunktion,
\begin{equation}
  f(x) = mx + b \label{linfit}
\end{equation}
die Kalibrationsgerade aus Abbildung \ref{plot:kali} gewonnen werden.
\begin{figure}
  \centering
  \includegraphics[height=8cm]{build/eichung.pdf}
  \caption{Kalibrationswerte und -gerade bestimmt über einen linearen Fit \ref{linfit}.}
  \label{plot:kali}
\end{figure}
Die Fitparameter lauten
\begin{align*}
  m &= \input{build/m_val.tex} \: \text{pro Kanal,}\\
  b &= \input{build/b_val.tex}.
\end{align*}
Über diese Kalibrationsfunktion können den aufgenommenen Messwerten, welche Kanäle zugeordnet wurden, nun Zeiten zugeordnet werden.

\subsection{Ergebnisse der Messung der Lebensdauer}

Zum Fitten der Zerfallsrate wird $curve_fit$ aus der python Bibliothek scipy.optimize mit der Fitfunktion
\begin{equation}
  N(t) = N_0 \exp{-\lambda t} + U_0 \label{exp}
\end{equation}
verwendet.
Der Fehler des Messwertes eines Kanals entpricht der Wurzel der gemessenen Anzahl. %but wryyyyyy?
Die Parameter werden auf die Werte \label{params}
\begin{align*}
  N_0 &= \input{build/N_0_val.tex},\\
  \lambda &= \input{build/lambd_val.tex},\\
  U_0 &= \input{build/U_val.tex}
\end{align*}
bestimmt, $U_0$ entspricht dabei dem Untergrund.
Das Ergebnis ist in Abbildung \ref{plot:punkte_err_fit} zu begutachten.
\begin{figure}
  \centering
  \includegraphics[height=8cm]{build/expfit.pdf}
  \caption{Aufgenommene Messwerte der Individuallebensdauern mit Fehlern und Fit der Funktion \ref{exp}.}
  \label{plot:punkte_err_fit}
\end{figure}
Die mittlere Lebensdauer errechnet sich damit zu
\begin{align*}
  \tau = \frac{1}{\lambda} = \input{tau.tex}.
\end{align*}
Zu Darstellungszwecken sind in Abbildung \ref{plot:punkte_sigma} die Messpunkte und das $2\sigma$-Intervall dargestellt.
\begin{figure}
  \centering
  \includegraphics[height=8cm]{build/expfit_sigma.pdf}
  \caption{Aufgenommene Messwerte der Individuallebensdauern und Funktion \ref{exp} mit den Parametern \ref{params} mit $2\sigma$-Intervall.}
  \label{plot:punkte_sigma}
\end{figure}


\subsection{Theoretische Berechnung des Untergrundes}

Der Untergrund wird dominiert durch ein zweites Myon, welches den Stopimpuls auslöst.
Dieser Störeffekt kann ausgerechnet werden, da die Ankunft der Myonen im Tank poissonverteilt ist.
Es werden $N_\text{ges} = \num{2882767}$ Myonen während der totalen Messzeit von $T_\text{ges} = \SI{160147}{\second}$ gemessen.
Während der Suchzeit $T_\text{S} = \SI{15}{\micro\second}$ kommen demnach durchschnittlich
\begin{align*}
  \lambda_{pv} = \frac{N_\text{ges}}{T_\text{ges}} T_\text{S}
\end{align*}
Myonen an.
Die Wahrscheinlichkeit, dass genau ein zweites Myon während der Suchzeit den Tank durchquert, beträgt
\begin{align*}
  P = \lambda_{pv} \exp{-\lambda_{pv}}
\end{align*}
multipliziert mit der gesamten Anzahl an gemessenen Myonen ergibt sich ein Untergrund von
\begin{align*}
  U_{\text{Theorie}} = \input{build/U_theo.tex}.
\end{align*}
