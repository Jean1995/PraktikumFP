\section{Auswertung}
\label{sec:Auswertung}
\subsection{Bestimmung der Lage des Intensitätsmaximums ohne Magnetfeld}

Zunächst wird die Intensitätsverteilung im Detektor ohne Magnetfeld, d.h. ohne Ablenkung des Teilchenstrahls bestimmt.
Dabei wird, um die Position des Detektors aus der Lage des Potentiometers zu ermitteln, der Umrechnungsfaktor $\SI{1.8}{\milli\metre}$ verwendet.
Die Spannung des Detektors wird gegen seine Lage $l$ aufgetragen, was in Abbildung \ref{plot0} graphisch dargestellt wird.
\begin{figure}
  \centering
  \includegraphics[height=5cm]{build/plot0.pdf}
  \caption{Messung der Intensitäten ohne Magneteld.}
  \label{fig:plot0}
\end{figure}
Um die Lage des Maximums exakt zu ermitteln, wird ein Fit an eine Gaußfunktion
\begin{equation}
  a \exp{\left( -\frac{1}{2} \frac{l - \mu}{\sigma**2} \right)} + b
\end{equation}
durchgeführt.
Der Fit wird dabei mithilfe von SciPy von Python durchgeführt.
Aus dem Fit ergibt sich
\begin{align*}
  \mu = \input{build/s_mitte.tex},
\end{align*}
was der Lage des Intensitätsmaxiums entspricht.

\subsection{Bestimmung der Lagen der Maxima mit Magnetfeld}
Im Folgenden wird das Magnetfeld eingeschaltet, so dass eine Ablenkung stattfindet.
Die Umrechnung von der Stromstärke der Spule $I$ in das Magnetfeld $B$ folgt über ein dem Verusch beiliegendem Diagramm.
Für die verschiedenen Magnetfelder werden zunächst die Spannungen des Detektors gegen seine Lage abgetragen.
Daraufhin wird jeweils ein Fit an die Summe zweier Gaußfunktionen,
\begin{equation}
  a \exp{\left( -\frac{1}{2} \frac{l - \mu_1}{\sigma_1**2} \right)} +  b \exp{\left( -\frac{1}{2} \frac{l - \mu_2}{\sigma_2**2} \right)} + c
\end{equation}
durchgeführt.
Somit entsprechen $\mu_1$ und $\mu_2$ den Lagen der beiden Intensitätsmaxima.
Die Ergebnisse der Fits sind in den Abbildungen \ref{fig:fit1} bis \ref{fig:fit8} dargestellt.

\begin{figure}
  \centering
  \includegraphics[height=5cm]{build/plot1.pdf}
  \caption{Messung der Intensitäten mit Magnetfeld für $I=\SI{0.3}{\ampere}$.}
  \label{fig:plot1}
\end{figure}

\begin{figure}
  \centering
  \includegraphics[height=5cm]{build/plot2.pdf}
  \caption{Messung der Intensitäten mit Magnetfeld für $I=\SI{0.4}{\ampere}$.}
  \label{fig:plot2}
\end{figure}

\begin{figure}
  \centering
  \includegraphics[height=5cm]{build/plot3.pdf}
  \caption{Messung der Intensitäten mit Magnetfeld für $I=\SI{0.5}{\ampere}$.}
  \label{fig:plot3}
\end{figure}

\begin{figure}
  \centering
  \includegraphics[height=5cm]{build/plot4.pdf}
  \caption{Messung der Intensitäten mit Magnetfeld für $I=\SI{0.6}{\ampere}$.}
  \label{fig:plot4}
\end{figure}

\begin{figure}
  \centering
  \includegraphics[height=5cm]{build/plot5.pdf}
  \caption{Messung der Intensitäten mit Magnetfeld für $I=\SI{0.7}{\ampere}$.}
  \label{fig:plot5}
\end{figure}

\begin{figure}
  \centering
  \includegraphics[height=5cm]{build/plot6.pdf}
  \caption{Messung der Intensitäten mit Magnetfeld für $I=\SI{0.8}{\ampere}$.}
  \label{fig:plot6}
\end{figure}

\begin{figure}
  \centering
  \includegraphics[height=5cm]{build/plot7.pdf}
  \caption{Messung der Intensitäten mit Magnetfeld für $I=\SI{0.9}{\ampere}$.}
  \label{fig:plot7}
\end{figure}

\begin{figure}
  \centering
  \includegraphics[height=5cm]{build/plot8.pdf}
  \caption{Messung der Intensitäten mit Magnetfeld für $I=\SI{1.0}{\ampere}$.}
  \label{fig:plot8}
\end{figure}
