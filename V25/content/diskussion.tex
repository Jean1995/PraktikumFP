\section{Diskussion}
\label{sec:Diskussion}

Die Ergebnisse für das Bohrsche Magneton aus der durchgeführten Rechnung lauten
\begin{align*}
  \mu_{\text{B,F,mittel}} &= \input{build/mu_f_mittel.tex},\\
  \mu_{\text{B,H,mittel}} &= \input{build/mu_h_mittel.tex}.
\end{align*}
Verglichen mit dem Literaturwert \cite{Konstanten}
\begin{align*}
  \mu_{\text{B,lit}} &= \input{build/mu_lit.tex}
\end{align*}
entspricht dies Abweichungen von
\begin{align*}
  \delta \mu_{\text{B,F,mittel}} &= \input{build/mu_f_abw.tex},\\
  \delta \mu_{\text{B,H,mittel}} &= \input{build/mu_h_abw.tex}.
\end{align*}
Dieses Ergebnis zeigt signifikante Abweichungen von Literaturwert nach unten, liegt trotz allem noch in der gleichen Größenordnung.
Bei der Bewertung der Ergebnisse sind vor allem zwei entscheidende Faktoren zu nennen.
Der erste Faktor ist der Sachverhalt, dass zur Ermittlung des Magnetfeldes $B$ ein vorgegebener Graph verwendet wurde.
Einerseits ist das Ablesen dieses handgeschriebenen Graphen über Millimeterpapier relativ ungenau, andererseits sind die verwendeten Ergebnisse zum Zeitpunkt der Versuchsdurchführung über 15 Jahre alt gewesen.
Ob die Daten somit nach wie vor gültig sind kann somit hinterfragt werden.
Eine signifikante Verbesserung der Umrechnung wäre bei genauer Kenntnis der Permeabilitätszahl $\mu_r$ des verwendeten Magneten möglich, da dann ausgehend von der bekannten Geometrie das Magnetfeld $B$ aus $I$ ermittelt werden könnte.

Der zweite Faktor ist die Tatsache, dass das Ablesen der Maxima schwierig war.
Auch wenn die Bestimmung der Maxima über Fits bei erster Betrachtung ungenauer schienen, sind die daraus folgenden Ergebnisse für $\mu_\text{B}$ genauer als die Ergebnisse, wenn die Maxima per Hand abgelesen werden.
Eine Verbesserung der Fitmethode bzw. die Verwendung einer anderen Fitfunktion würde das Ergebnis weiterhin verbessern.

Auch weitere Faktoren wie beispielsweise ein nicht vollständiges Vakuum oder die ungenaue Bewegung des Detektors führen zu Ungenauigkeiten.
Letzteres hat zu einem systematischen Fehler geführt, welcher darin resultierte, dass die Messergebnisse aufgeteilt werden mussten.
Ohne diesen Fehler wäre eine höhere Genauigkeit möglich gewesen.
