\subsection{Durchführung}
\label{sec:durchführung}
Zunächst wird die Intensitätsverteilung ohne Magnetfeld gemessen.
Vorbereitend wird hierfür der Kaliumofen eine Stunde vorgeheizt und die Vakuumpumpe eingeschaltet, bis ein gewünschtes Vakuum erreicht wird.
Ist die Endtemperatur des Ofens erreicht, kann mit der Messung begonnen werden.
Hierfür wird die einfallende Spannung am LT-Detektor in Abhängigkeit der Stellung des Schwenkarms abgelesen.
Dieses geschieht für den gesamten Intensitätsbereich, wobei in der Nähe starker Änderungen geringere Abstände gewählt werden.
Während einer Messreihe ist es wichtig den Arm nur in eine Richtung zu schwenken, da sonst die relative Auslenkung des Armes nicht richtig gemessen werden kann.\\
Nach dieser Nullmessung, wird erneut der Intensitätsbereich für acht verschiedene Feldgradienten aufgenommen.
Dabei ist aufgrund von Hystereseeffekten zu beachten, das Magnetfeld zwischen den Messreihen nur zu erhöhen.
Zusätzlich wird die Temperatur des Ofens bei jeder Messung aufgenommen.
